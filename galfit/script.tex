\documentclass[preprint]{aastex}
\usepackage{emulateapj5}
\usepackage{amsmath}
\usepackage{apjfonts}
\usepackage{natbib}

\def\galfitcl{{\it galfit.cl}}
\def\galfitpl{{\it galfit.pl}}

\begin{document}

\slugcomment{For the GALFIT Fitting Script}

\title{GALFIT FITTING SCRIPT USER GUIDE}

\author {Chien Y. Peng\altaffilmark{1}}
\altaffiltext{1}{Steward Observatory, University of Arizona, 933 N. Cherry
Av., Tucson, AZ 85721;  cyp@as.arizona.edu.}

\begin {abstract}

This document is intended to explain how the two scripts \galfitcl\  and
\galfitpl, when used in conjunction with GALFIT, can be used to automate
galaxy fitting.  The IRAF script \galfitcl\  is a master wrapping script
that runs the Perl script, \galfitpl, which finally runs the C-program,
GALFIT.  How the scripts determine the initial parameters for GALFIT is based
on a SExtractor catalog.  Under normal operation, one should not have to touch
\galfitpl\  itself, and never has to run GALFIT manually.  One only has to
modify the parameters through IRAF then let \galfitcl\  do its job.

Having said that, this document, as well as the scripts, are still highly
experimental, and will be for the foreseeable future.  So they are not as
fine-tuned or user friendly compared to the main GALFIT program itself.  I
also don't know if there's a time when the script can be fully generalized and
user friendly.  So the usefulness of the scripts will rely in no small part on
user experience, and his/her willingness to ``hack'' into various parts of
\galfitcl\ and \galfitpl\ to suit his/her needs.  However, hopefully one has
to only tweak what's already provided.  Therefore, part of the reason for this
document is to point out where I think improvements can be made, and provide
some kind of guide to novice programmers or non-programmers who wish to modify
the scripts.

\end {abstract}

\section {INTRODUCTION}

GALFIT (the C-program) doesn't have an in-built way to fit galaxies in batch
mode autonomously.  It is not easy to write a program to consider all possible
complications in batch fitting.  Complications start almost from the very
beginning.  For example, how does one decide on the galaxies to fit, and how
does one obtain the initial parameters for GALFIT?  How should one deal with
neighboring galaxies:  does one fit them together using multiple components,
or does one mask out all galaxies except the principle galaxy from the fit?  A
program must also decide on an image size, which is complicated by neighbors
and surface brightness considerations.  Is one interested in bulge-to-disk
decomposition, or something more simple/complicated?  One code, GIM2D, has
tried to do many of this, by allowing the user plenty of parameters to adjust
depending on the data set.  Because of the generality, it is understandable
the learning curve can be a bit steep.  I take a different approach, which is
to provide core minimization algorithm (GALFIT), which the users might more
fully understand.  Then, it's up to the users to customize a wrapping script
around GALFIT for their own needs, as they find best.

A second reason why there has not been a standard batch fitting GALFIT code is
a justified concern about making a gradient-based algorithm fully automated.
Currently there is not any literature (that I know about) which shows how
robust a gradient algorithm is when used in a fully automated way.  If GALFIT
were to not converge on a global minimum $\chi^2$, it needs a human to guide
it along, which is not feasible when fitting a large sample of galaxies.  In
general, gradient-based algorithms are less robust in highly complex
situations when compared to another class of algorithms called the ``simulated
annealing'' or ``metropolis''.  Despite this fact, it is not obvious which
algorithm is the better choice because there are inherent subtleties in both
which one should be aware.  For a gradient algorithm, it is sometimes
important to start the parameters off in the right ``ball park,'' which may
sometimes be hard to determine in an automated way.  And, how sensitive are
the solutions to initial parameters?  On the other hand, in simulated
annealing, one has to consider how fast to decrease the annealing
``temperature''.  Cool too quickly, the solution may converge on a local
minimum; too slowly, and the solution bounces around the parameter space.
Hence, simulated annealing is very inefficient compared to a gradient
algorithm, even when fitting a single galaxy component.  It is also subtle how
this parameter depends on the image noise properties, complexities of the
galaxy environment, and the number of components being fitted.  So far there
hasn't been any direct comparison between simulated annealing and gradient
method when it comes to galaxy fitting.

To fit galaxies in batch mode in a fully automated way, we put GALFIT inside a
wrapping script that does all the bookkeeping before and after running GALFIT.
The users can then tailor and fine-tune the scripts for his/her own needs
without having to modify the central, minimization, engine.  The two scripts
\galfitcl\ (IRAF) and \galfitpl\ (Perl) found in the GALFIT package are the
first attempts at automating GALFIT for batch fitting.  We use two scripts to
takes advantage of the fact IRAF can easily deal with image headers and is
good for manipulating images, while Perl, which is superior at manipulating
strings and text files, does all the hard work.

During operations, one only runs the \galfitcl\ script.  One shouldn't have to
tweak the scripts \galfitpl\ or GALFIT itself.  The only reason why anyone
would want to play with \galfitpl\ is if one would like to play around with
the AI, or if one believes that the galaxy fitting boxes are too small, etc..


\section {THE IRAF SCRIPT {\it galfit.cl}}

\galfitcl\  is the top-level wrapping script, which is run in IRAF.  To
run it, it has to be declared first as an IRAF task.  For example, I do:

\footnotesize
\begin{verbatim}
cl> task galfit = "/home/cyp/iraf/scripts/galfit.cl"
\end{verbatim}
\normalsize

\noindent It then runs just like any other IRAF task, {\it epar, lpar} and
all.  (Note:  the default place is to put \galfitpl\  and \galfitcl\  in
the same directory.)

\galfitcl\ handles input parameters such as image name, SExtrator catalog,
output file names, and other parameters that one can not infer from the
SExtractor catalog.  This script is the only one the user has to interact with
in practice.  One of its important task of \galfitcl is to be a master
bookkeeper:  It divides up a big image into sub-tiles (to be discussed soon),
and keeps track of which galaxy belongs in which sub-tile.  After all the
objects are fitted, it figure out how to convert coordinates local to the
sub-tiles into the global coordinate.  \galfitcl\ also runs the Perl script
\galfitpl.

If one does an {\it lpar} or {\it epar} on "galfit" in IRAF, one would see
something akin to Figure 1.  The meaning of the parameters are the following:

\begin {itemize}

\item image -- the input data image

\item sigimg -- the sigma image, i.e. the variance at each pixel which is used
to calculate the $\chi^2$ statistics and the parameter uncertainties.

\item sexcat -- the SExtractor catalogue name, which provides initial
parameters for fitting for all galaxies.  The output columns from SExtractor
are shown in Fig. 2.  Please see the SExtractor manual on how to use that
software.

\item startnum and stopnum -- lets you choose which objects you want to
fit, in case you don't want to go through the whole catalogue.

\item psfloc -- the directory where all the PSFs are stored.  The PSF have
have names: psf{\it string}-{\it xxxx}-{\it yyyy}.fits where {\it xxxx} and
{\it yyyy} are the pixel coordinates of the PSF.  For example, the PSF can be
named "psfz-925-218.fits".  This is useful if the PSF is field dependent, and
one would like to use a PSF nearest to a galaxy.

\item constraint -- the parameter constraint file used in GALFIT.

\item fitfunc -- single S\'ersic or B/D decomposition.

\item region -- the image region region to fit.  This is useful if one
wants to to fit a subsection of the image.

\item boundary -- the region to fit if one wants to fit an image subsection.
otherwise it is ignored.

\item skyval -- the difference in the sky value of the data from the
SExtractor measured sky value.  This is sometimes needed because SExtractor
gets run on an image with sky subtracted out.  But, because GALFIT needs the
real sky to calculate $\chi^2$ when the sigma image is not provided, sometimes
people would add the sky value back into their data, and not the SExtractor
catalog.  So then, the actual starting parameter for the sky value when GALFIT
starts fitting should be: {\it sky = skyval + SExtractor sky}, where {\it
skyval} is the value provided here, and {\it SExtractor sky} is that found in
the catalogue.

\item mrange -- objects within the magnitude range are fitted as the primary
object.

\item stargal -- objects that are classfied as star/galaxy within these
numbers, based on the SExtractor classifier index, are fitted.

\item fwhmpsf -- the FWHM of the PSF, for rejecting cosmic rays.

\item magzpt -- the photometric zeropoint for GALFIT.

\item plate -- the plate scale for GALFIT.  (Not sure why this is needed.)

\item exist -- delete the existing sub-tile images and restart from scratch?

\item nofat -- delete all the working files, including the sub-tiles, after
fitting?

\item nsplit -- the number of subtiles to break the original image into.  If
the fitting image is too large, one can have the script break it up into
subtiles so GALFIT doesn't have to read in a huge image each time for each
galaxy.  This can save a whole lot of time and memory, as GALFIT stores the
whole image in memory as it works on it.  This script keeps track of all the
accounting needed to convert the coordinates back to the original data image,
so one doesn't have to worry about converting all the coordinates back
manually.  The final coordinates relative to the master tile are stored in
the image [2] of the galfit output block.

\end {itemize}

This should be enough to get the user off and running, once the user has the
SExtractor output file for the "sexcat" parameter.  Figure 2 shows what
columns are needed to be in the SExtractor catalog.  For discussions on how to
output those parameters, and what they mean, please see the SExtractor manual.
Not all the parameters shown in Figure 2 are actually useful for our purposes.
They are used as part of the GEMS survey, so unless one is willing to modify
the scripts of \galfitcl\ and \galfitpl, one is stuck with it for now.

\newpage
\footnotesize

\begin {verbatim}
        image = "s1z09.fits"    Input image to be fitted
       sigimg = ""              Sigma image to use
       sexcat = "s1z09.wht.cat" Sextractor catalog name
     startnum = 20              Start from which object number in the catalog?
      stopnum = 999             Stop at which number in the catalog?
      (psfloc = "/home/cyp/gems/47tuc/psfs-0.05/zpsfs") PSF location (directory)
  (constraint = "constraints.use") Parameter constraint file
     (fitfunc = "BD")           Single sersic or B/D decomposition?
      (region = "A")            (A) Whole image catalog or, (B) region of image
    (boundary = "")             Boundary of region if (B) (xmin xmax ymin ymax)
      (skyval = 14.5)           Sky background to add back to Sextractor value
      (mrange = "0 24")         Acceptable magnitude range
     (stargal = "0. 0.7")       Range of good galaxies: 0.0 (gal) to 1.0 (star)
     (fwhmpsf = "0.06")         FWHM of the PSF for cosmic ray rejection
      (magzpt = "25.978")       Photometric zeropoint
       (plate = "0.05")         Image plate scale
       (exist = no)             Rid all existing temp-*.fits images before run?
       (nofat = no)             Rid all temp images, galfit input, after run?
      (nsplit = 10.)            Split image into how many parts along one axis?

\end {verbatim}
\figcaption[]{Example of parameters for galfit.cl IRAF task.}


\begin {verbatim}
#   1 NUMBER          Running object number
#   2 FLUX_BEST       Best of FLUX_AUTO and FLUX_ISOCOR               [count]
#   3 FLUXERR_BEST    RMS error for BEST flux                         [count]
#   4 MAG_BEST        Best of MAG_AUTO and MAG_ISOCOR                 [mag]
#   5 MAGERR_BEST     RMS error for MAG_BEST                          [mag]
#   6 KRON_RADIUS     Kron apertures in units of A or B
#   7 BACKGROUND      Background at centroid position                 [count]
#   8 ISOAREA_IMAGE   Isophotal area above Analysis threshold         [pixel**2]
#   9 X_IMAGE         Object position along x                         [pixel]
#  10 Y_IMAGE         Object position along y                         [pixel]
#  11 ALPHA_J2000     Right ascension of barycenter (J2000)           [deg]
#  12 DELTA_J2000     Declination of barycenter (J2000)               [deg]
#  13 THETA_IMAGE     Position angle (CCW/x)                          [deg]
#  14 ELLIPTICITY     1 - B_IMAGE/A_IMAGE
#  15 FWHM_IMAGE      FWHM assuming a gaussian core                   [pixel]
#  16 FLAGS           Extraction flags
#  17 CLASS_STAR      S/G classifier output
\end {verbatim}
\figcaption[] {Output columns in the SExtractor catalog.}

\newpage

\end {document}
