\documentclass[preprint]{aastex}
\usepackage{emulateapj5}
\usepackage{amsmath}
\usepackage{apjfonts}
\usepackage{natbib}

\begin{document}

\slugcomment{For GALFIT 2.0 and more recent versions}

\title{GALFIT QUICK START GUIDE}

\author {Chien Y. Peng\altaffilmark{1}}
\altaffiltext{1}{Space Telescope Science Institute, 3700 San Martin Drive, 
Baltimore, MD 21218;  cyp@stsci.edu.}

\begin {abstract}

This document is intended to get new users quickly started on running GALFIT.
For a more detailed idea of how the program is implemented, see {\it Peng, Ho,
Impey, \& Rix (2002, \aj, 124, 266).} There have been a number of upgrades
since the publication of Peng et al. (2002), improving on techniques, adding
new features, and fixing known bugs.  Therefore, this document supersedes that
AJ article whenever there are differences.

\end {abstract}

\section {INTRODUCTION}

GALFIT is an algorithm that analyzes the light profile of astronomical objects
by fitting/decomposing them with one or more analytic functions.  Each
function has a set of adjustable free parameters that are tuned to match the
light distribution of an image, and multiple functions can be fitted
simultaneously.  The available functional forms are defined below.  Though the
original purpose was to fit galaxies, this algorithm can be used in a broader
context of morphology studies or for deblending objects into one or more
components.  For example, one can analyze images of globular clusters,
circumstellar disks, star clusters, galaxy bars, and AGNs.  GALFIT is fairly
fast and flexible so it is useful for analyzing complex images manually,
quickly, and robustly.  It is also modular, allowing for the possibility to
carry out large surveys where the fitting is done in a fully automated manner.
This involves inserting GALFIT into a ``wrapping script'', and an example is
provided in the software package (see the {\it galfit.pl} and {\it galfit.cl}
tasks).

To find the best fit, GALFIT minimizes the $\chi^2$ residual between the data
image and the model, adjusting all the free parameters simultaneously.  This
is done by using a Levenberg-Marquardt algorithm, which is modified from the
Numerical Recipes (Press et al.  1997) subroutine.  This
downhill-gradient-type algorithm is among the fastest for searching large
parameter spaces.  The definition of $\chi^2$ is:

\begin {equation}
\chi^2 = \sum_{x=1}^{nx}\sum_{y=1}^{ny} \frac {\left(\mbox{flux}_{x,y} - \mbox{model}_{x,y}\right)^2} {{\sigma_{x,y}}^2},
\end {equation}

\noindent summed over all $nx$ and $ny$ pixels.  $\sigma_{x,y}$ is the weight
at each pixel, for all pixels.  The weight or ``sigma'' at each pixel is
either computed by GALFIT automatically based on Gaussian (Poisson) noise, or
can be supplied by the user in the form of an image, which I call the ``sigma
image''.

The fitting functions available in GALFIT are:  S\'ersic/de~Vaucouleurs,
Nuker, exponential, Moffat, Gaussian, (modified) King, and the PSF.  All of
the models, except the PSF, are convolved with a PSF to simulate the blurring
caused by the telescope optics and/or the atmosphere.  If a PSF is not
provided fitting can still proceed without PSF convolution.  As of version
2.0, GALFIT allows the PSF to be more finely sampled than the actual science
data.  This is a useful feature for HST data whereby a super-sampled PSF may
be generated by TinyTim (Krist \& Hook 1997) or other methods.

\subsection {Compiling the Source Code}

To compile GALFIT, please follow the procedure in the README.INSTALL file on
the GALFIT web-homepage or in the software package.

\subsection {Important FITS Header Parameters and the Weight (Sigma) Image}

GALFIT looks for four standard header keywords {\it usually} found in a FITS
image: EXPTIME, GAIN (or ATODGAIN), RDNOISE, and NCOMBINE.  These are,
respectively, the image exposure time, gain (in units of electrons/ADU), image
readnoise (in units of electrons), and the number of combined images.  If
these keywords are not found for some reason, GALFIT assumes the default
values of 1, 7, and 5.2, 1, respectively (which correspond to values for the
PC camera on HST).

EXPTIME is used to calculate the object magnitude and surface brightness of
the fit.  RDNOISE, GAIN, and NCOMBINE are used only to calculate a sigma map
(i.e. a weight map, i.e. $\sigma_{x,y}$ in Eq. 1 above) to weigh the pixels in
the fit based on the Poisson noise model.  Alternatively, the user may supply
her own sigma image (see \S~3.4.1), whereupon the RDNOISE, GAIN, and NCOMBINE
parameters do not matter.

{\it To get the sigma image, the default assumption in GALFIT is that the
RDNOISE and GAIN are values for a single readout and the image is {\it
averaged} over NCOMBINE images (not summed)}.  As GALFIT converts the image
ADUs into electrons using those parameters, the ADU unit should be in {\it
counts} -- not counts/sec, not electrons (unless, of course, GAIN=1), and not
electrons/sec.  On the other hand, if the image is formed by summing
sub-exposures, the equivalent is to set NCOMBINE $= 1$, keeping RDNOISE and
GAIN for also a {\it single} readout.

Sometimes the sky background in a reduced science image has already been
removed during data reduction.  If the user does not provide a sigma image,
GALFIT will attempt to do so automatically.  With the sky removed, the
statistics calculation for low signal-to-noise objects may not be accurate.
The original background sky level should thus be added back if it has been
subtracted.  If the DC level that has been removed is not known, one can
nonetheless ``reverse engineer'' what the sky level was originally from the
sky noise, and add that DC level back to the image.

% In GALFIT, the empirical sigma image is generated by first convolving the
% image with a Gaussian Kernel of $\sigma=2$ pix and radius of 3 pixels.  The
% effective gain at each pixel is $gain_{eff}$ = NCOMBINE $\times$ GAIN, and the
% effective readnoise is: $\sigma_{rd}$ = RDNOISE $\times
% \sqrt{\mbox{NCOMBINE}}/gain_{eff}$ [ADUs].  If the pixel value is less than
% one electron, the noise is assumed to be one electron.  The poisson noise of
% the flux is thus $\sigma_p = \sqrt {f_{pix} * gain_{eff}}/gain_{eff}$ [ADUs],
% where $f_{pix}$ is the smoothed flux at each pixel.  The sigma map is obtained
% by adding the two noise terms:  $\sigma_p$ and $\sigma_{rd}$ in quadrature.
% Again, this process is skipped if the user provides own input sigma image.

\subsection {Quick Start: Running the GALFIT Example}

The quickest and easiest way to run GALFIT is to have a pre-formatted template
file like the one shown in Figure 1 (also look for the EXAMPLE.INPUT file in
the GALFIT source directory).  Skip to \S~4 to see what the menu parameters
mean.  In the ``galfit/EXAMPLE'' directory there is a simple example that one
can try out immediately.  After GALFIT has been compiled and aliased/linked,
to run the example, go into that directory and simply type ``galfit
galfit.feedme''.  After GALFIT finishes running it will produce three output
files called ``galfit.01'', ``fit.log'', and ``imgblock.fits''.  To understand
these files, please see Section 6.

You now know how to run GALFIT.  The rest of this document will explain what
the parameters in the input file mean and the output files.

\bigskip

\noindent {\bf Please also read the ``Frequently Asked Questions and
Advisory'' at the end of this document.}

\section {FUNCTIONAL FORMS}

In this section, I will briefly describe the analytic functions available in
GALFIT for fitting light profiles.  

\subsection {Generalized Ellipsoid Parameters}

All the profiles are axisymmetric, generalized ellipses in shape (i.e.
azimuthally).  Thus, the (elliptical) radial distance of a pixel ($x,y$) from
an object at ($x_c, y_c$) is defined as:

\begin {equation}
r = \left(\left|x-x_c\right|^{c+2}+\left| \frac{y-y_c}{q}\right|^{c+2}\right)^{\frac{1}{
c+2}}.
\end{equation}

\noindent Here, the ellipse axes are aligned with the coordinate axes.  The
ellipse is general in the sense that $c$ is a free parameter, which controls
the diskiness/boxiness of the isophote.  When $c=0$ the isophote shape is a
pure ellipse, but becomes increasingly disky (diamond-like) with decreasing
$c$ ($c < 0$) and boxy (rectangular) as $c$ increases ($c>0$).  The free
parameters in Eq. 2 are the axis ratio ($q$), and the diskiness/boxiness
parameter ($c$) and the centroid of the ellipsoid, ($x_c$ and $y_c$).  The
major axis of the ellipse can also be oriented to a position angle (PA -- not
shown).  Thus, there are a total of 5 free parameters.

\subsection {The Radial Profiles}

The radial profile can have the following functional forms.  These are some of
the most frequently seen in literature, and more will likely be added in the
future.  If there is any other which you are interested in fitting, feel free
to contact me. 

\bigskip
\noindent {\bf The S\'ersic Profile}\ \ \ \ \ The S\'ersic power law is one of
the most frequently used to study galaxy morphology, and has the following
functional form:

\begin {equation}
\Sigma(r)=\Sigma_e \exp{\left[-\kappa \left(\left({\frac{r}{r_e}}\right)^{1/n} - 1\right)\right]}.
\end {equation}

\noindent $\Sigma_e$ is the pixel surface brightness at the effective radius
$r_e$.  The parameter $n$ is often referred to as the concentration parameter.
When $n$ is large, it has a steep inner profile, and a highly extended outer
wing.  Inversely, when $n$ is small, it has a shallow inner profile and a
steep truncation at large radius.  The parameter $r_e$ is known as the
effective radius such that half of the total flux is within $r_e$.  To make
this definition true, the dependent-variable $\kappa$, is coupled to $n$, thus
it is not a free parameter.  The classic de~Vaucouleurs profile that describes
a number of galaxy bulges is a special case of the S\'ersic profile when $n$ =
4 (thus $\kappa = 7.67$).

The flux integrated out to $r=\infty$ for a S\'ersic profile is:

\begin{equation} 
F_{\rm tot} = 2\pi r_e^2 \Sigma_e {\mbox e}^\kappa n \kappa^{-2n}\Gamma(2n) q/R(c),
\end{equation}

\noindent where

\begin{equation}
R(c) = {\frac{\pi (c+2)} {4 \beta (1/(c+2), 1+1/(c+2))}}.
\end{equation}

\noindent $\beta$ is the so-called Beta function.

In GALFIT, the flux parameter that is fitted for the S\'ersic function is the
{\it integrated magnitude} ($m_{tot}$), not $\Sigma_e$ (the user may do the
inversion based on Eq. 4 and 5 to get $\Sigma_e$).  The integrated magnitude
is the standard definition:

\begin {equation}
m_{tot} = -2.5  {\mbox {log}_{10}} \left( \frac{F_{tot}}{t_{exp}}\right) + {\mbox {mag zpt}},
\end {equation}

\noindent where $t_{exp}$ is EXPTIME from the image header.  Each S\'ersic
function can thus potentially have 8 free parameters in the fit:  $x_c$,
$y_c$, $m_{tot}$, $r_e$, $n$, $q$, PA, $c$.

\bigskip

\noindent{\bf The Exponential Profile}\ \ \ \ \ The exponential profile is a
special case of the S\'ersic function for when $n=1$.  And instead of using
the effective radius $r_e$ to characterize an object size, here the
scalelength ($r_s$) is used.  When the S\'ersic index $n=1$, $r_e = 1.678
r_s$.  The functional form is:

\begin {equation}
\Sigma(r)=\Sigma_0 \exp{\left(-\frac{r}{r_s}\right)}
\end{equation}
\begin{equation}
F_{\rm tot} = 2\pi r_s^2 \Sigma_0 q/R(c),
\end {equation}

\noindent The free parameters of the profile are the total magnitude and
$r_s$, in addition to the parameters for the generalized ellipse,
for a total of 7 free parameters.

\bigskip

\noindent{\bf The Gaussian Profile}\ \ \ \ \ The Gaussian profile is another
special case of the S\'ersic function for when $n=0.5$, but here the size
parameter is the FWHM instead of $r_e$.  The functional form is:

\begin{equation}
\Sigma(r) = \Sigma_0 \exp{\left(\frac{-r^2} {2 \sigma^2}\right)}
\end{equation}
\begin{equation}
F_{\rm tot} = 2\pi\sigma^2\Sigma_0 q/R(c),
\end {equation}

\noindent where FWHM = 2.354$\sigma$.  The free parameters of the profile are
the total magnitude and the FWHM, in addition to the ellipsoid parameters, for
a total of 7 free parameters.

\bigskip

\noindent{\bf The Empirical (Modified) King Profile}\ \ \ \ \ The empirical
king profile is often used to fit the light profile of globular clusters.  It
has the following form (Elson XXXX):

\begin{equation}
\Sigma(r) = \Sigma_0 \left[\frac{1}{(1+(r/r_c)^2)^{1/\alpha}} - \frac{1}{(1+(r_t/r_c)^2)^{1/\alpha}} \right]^\alpha.
\end{equation}

\noindent The standard empirical King profile has a powerlaw $\alpha= 2$.  In
GALFIT, $\alpha$ can be a free parameter.  In this model, the flux parameter
to fit is the central surface brightness (limit as $r_t/r_c
\rightarrow\infty$) $\Sigma_0$, but expressed in mag/arcsec$^2$ form, i.e.
$\mu_0$, where

\begin{equation} 
\mu_0 = -2.5\ \rm{log}_{10} \left(\frac{\Sigma_0}{t_{exp} \Delta x \Delta y}\right) + \rm{mag\ zpt},
\end{equation}

\noindent where $t_{exp}$ is the exposure time from the image header, and
$\Delta x$ and $\Delta y$ are the platescale in arcsec, which the user
supplies (Item K in the GALFIT input file).  The other free parameters are the
core radius ($r_c$) and the truncation radius ($r_t$), in addition to the
geometrical parameters.  Outside the truncation radius, the function is set to
0.  Thus the total number of potential free parameters is 9.

\bigskip

\noindent {\bf The Nuker Law}\ \ \ \ \ The Nuker Law was introduced by Lauer
et al. (1995) to fit the nuclear profile of nearby galaxies, and it
has the following form:

\begin{equation}
I(r) = I_b \ 2^{\frac{\beta - \gamma} {\alpha}}
\left({\frac{r}{r_b}}\right)^{-\gamma}\left[{1+\left(\frac{r}
{r_b}\right)^{\alpha}} \right] ^{\frac{\gamma-\beta}{\alpha}}
\end{equation}

\noindent There are 5 free profile parameters: $\mu_b, r_b, \alpha, \beta,
\gamma,$ in addition to 5 ellipsoid parameters (see above).  The flux
parameter to fit is $\mu_b$, the surface brightness of the profile at $r_b$,
which is defined as:

\begin{equation}
	\mu_b = -2.5\ \rm{log}_{10} \left(\frac {I_b} {t_{exp} \Delta x \Delta y}\right) + \rm{mag\ zpt}
\end{equation}

\noindent The Nuker profile is a double powerlaw, where (in Eq. 13) $\beta$ is
the outer power law slope, $\gamma$ is the inner slope, and $\alpha$ controls
the sharpness of the transition.  The motivation for using this profile is
that many galaxies appear to be fit by this pretty well in 1-D (see Lauer et
al.  1995).  In all there are 10 free parameters.

\bigskip
\noindent{\bf The Moffat Function}\ \ \ \ \ The Moffat function is a special
case of the King profile when $r_t=0$ and when $\alpha = 1/n$.  The profile of
the HST WFPC2 PSF is also well described by the Moffat function.  Other than
that, the Moffat function is less frequently used than the above functions.
The functional form is:

\begin{equation}
\Sigma(r) = {\frac{\Sigma_0} {\left[1+(r/r_d)^2\right]^n}},
\end{equation}

\begin {equation}
F_{\rm tot} = {\frac {\Sigma_0 \pi r_d^2 q} {(n-1) R(c)}},
\end {equation}

\noindent The free parameters are the $m_{tot}$ (instead of $\mu_0$ as in the
King profile), FWHM (instead of $r_d$), the concentration index $n$, and
the 5 ellipsoid parameters.

\bigskip

\noindent{\bf Fitting the PSF} \ \ \ \ \ As of GALFIT version 1.9a, one can
fit pure stellar PSFs to an image (as opposed to narrow functions convolved
with the PSF).  The PSF function is simply the convolution PSF image that the
user provides (in Item D of the GALFIT menu), hence there is no prescribed
analytical functional form.  This is also the only profile that is not
convolved.  The PSF has only 3 free parameters: $x_c, y_c$ and total
magnitude.  Because there is no analytical form, the total magnitude is
determined by integrating over the PSF image and assuming that it contains
100\% of the light.  If the PSF wing is vignetted, there will be a systematic
offset between the flux GALFIT reports and the actual value.

If one wants to fit this ``function'', make sure the input PSF is close to, or
super-, Nyquist sampled.  The PSF shifting is done by something called the
Sinc+Kaiser interpolation, which can preserve the widths of the PSF even under
sub-pixel shifting.  This is in principle much better then Spline
interpolation or other high order interpolants.  However, if the PSF is
under-sampled, aliasing will occur, and the PSF interpolation will be poor.
If the PSF is undersampled, it is better to provide an oversampled PSF if
possible, even if the data is undersampled.  With HST data this can be done
using TinyTim or by combining stars.  GALFIT will take care of rebinning
during the fitting.

Note that the alternative to fitting a PSF is to fit a Gaussian with a small
width, i.e. 0.4-0.5 pixels, which GALFIT will convolve with the PSF.  This is
generally not advisable if a source is a pure point source because convolving
a narrow function with the PSF will broaden out the overall profile, even if
slight.  The convergence can also be poor if the FWHM parameter starts
becoming smaller than 0.5 pixel.  However, this technique can still be useful
to see if a source is truly resolved.

\bigskip

\noindent{\bf The Background Sky.}\ \ \ \ \ As of GALFIT version 1.8b, the
background sky is a flat plane that can tilt in x and y.  Thus it has a total
of 3 free parameters.  The pivot point for the DC sky level is taken to be at
geometric center $(x_0, y_0)$ of the image, calculated by $(n_{pix} + 1)/2$,
where $n_{pix}$ is the number of pixels along one dimension.  This position is
fixed.  The tip and tilt are calculated relative that center.  Because the
galaxy centroid located at $(x,y)$ is in general not at the geometric center
$(x_0, y_0)$ of the image, the sky value directly beneath the galaxy centroid
is calculated by:

\begin {equation}
    {\rm sky}(x,y) = {\rm sky} (x_0,y_0) + (x - x_0){{d\rm sky}\over{dx}} + (y - y_0) {{d\rm sky}\over{dy}}
\end {equation}

\bigskip

\section {RUNNING GALFIT}

Running GALFIT is easy, and so is quitting GALFIT.  GALFIT doesn't care when
and how you quit.  To quit abruptly, hit {\it control-c} at anytime at your
pleasure, but note that the results will not be saved.  In this section, I
will describe one of several ways to run (and quit) GALFIT.

\subsection {The Template File}

GALFIT is completely menu driven, and the menu can either come via a text file
template (easy! -- this is the way to go), or it can be filled in manually
(tedious! -- {\it not} a good way to go).  The menu items will be more
fully described in \S~4.

Figure 1 below shows an example of the GALFIT input (template) file.  When you
start GALFIT without a template file, you see a similar screen except the
entries are blank.  You can enter everything interactively on the command line
of GALFIT -- this is extremely tedious, but that's certainly possible (\S~3.3
will show you how).  After the menu Item S, the length of the list is
flexible, and it depends on how many components you want to fit.  You can add
or remove the number of components as you deem necessary.

In the input file, things after hash marks ``\#'' are comments, and are always
ignored by the program.  Blank lines are also ignored, and the column
alignment you see is optional (purely aesthetic).  There is pretty much no
error checking to catch problems in the input file, so I suggest one sticks to
the following format pretty closely.  For example, in the input file, do not
modify ``3) xxx'' (note the spaces), to look like ``3 )xxx'' (bad spacing).
Everything else about the format should be pretty intuitive.  If there are
errors in the input file, GALFIT may not complain about them.

\subsection{The Easiest Way to Run GALFIT: Reading in a Template File}

\subsubsection {From the Unix/Linux Prompt}

To facilitate fitting with minimal interaction, once you have a pre-formatted
text file shown in Figure 1, the typical GALFIT session is just a single
step.  On the Unix command line, type:

\begin{verbatim}
> galfit input_file_name
\end{verbatim}

If Item S is set to 1, GALFIT will wait for you to hit `q' before starting to
fit.  Otherwise it would start fitting right away.

The example shown below fits a galaxy with a PSF, a S\'ersic, exponential
disk, and a Nuker model, while holding the sky level constant at 2 counts
(ADU).  

\subsubsection {From Within GALFIT Prompt}

The second way to run GALFIT is by typing at the UNIX/LINUX command prompt:

\begin{verbatim}
> galfit
\end{verbatim}

\noindent If you don't have an input file, ignore the first question by
hitting return, which then brings you to the GALFIT prompt where you can edit
object and image parameters manually.  The GALFIT prompt looks like:

\footnotesize
\begin{verbatim}
Enter item, initial value(s), fit switch(es) ==>
\end{verbatim}
\normalsize

\noindent Below, I will not bother to re-show the prompt; all commands
issued are assumed to come after that generic prompt.

At the GALFIT prompt, you can read in one or more template files.  To do so,
simply type:

\begin {verbatim}
t file_name
\end {verbatim}

\noindent or 

\begin {verbatim}
t
\end{verbatim}

\noindent followed by a return.  Be careful about reading in two or more input
files:  while additional models will be {\it added} to the pre-existing file,
the parameters A-S, i.e. input data image, output file name, noise file, etc.,
will take on values of the newest input file.  (Note:  most of the time you
won't use this option anyway, but you can.)

\subsection {OPTIONAL Interactive Command Line Options While in GALFIT --
a.k.a.  ``Running GALFIT the Hard Way''}

\noindent{\it Note: you may skip this entire subsection and go on to \S~4 if
you already prepared your own input file.}

You can modify any of the parameters while on the GALFIT command line.  But,
you probably don't want to do this because the easiest way is to modify a
parameter file instead (which you can reuse should there be a crash), and feed
it into GALFIT on the Unix command line, as in \S 3.2.1.  The only time you
might want to do things interactively in GALFIT is if there's a strange
reason the input file isn't working.

\bigskip

\noindent {\bf (Re)displaying the Menu} \ \ \ \ \ To display or redisplay the
menu after new changes have been made, hit ``r''.

\bigskip

\noindent {\bf Adding New Objects, Changing Initial Parameters, Deleting
Objects} \ \ \ \ \ If you don't have an input file when you first start
GALFIT, hit return when the program asks you for one.  Then you can enter
initial parameters, add new objects, etc., interactively on the command line.

To add a new object on the command line, hit 0 or N, followed by the name of
the model you want to add.  For example, when you get the GALFIT prompt,
simply type:

\begin {verbatim}
0 devauc
\end {verbatim}

\noindent will add a de~Vaucouleurs function.  Initially, all the parameters
are set to 0 which you can then change.

To change the value of an item, enter the following on the GALFIT command
line in consecutive order, separated by one or more spaces only:

\begin{enumerate}
  \item the item number/alphabet (without the right parenthesis), 
  \item the initial value, then 
  \item optionally followed by whether to hold that parameter fixed or not
     during the fit.  To hold a parameter fixed, use the value 0; to fit, use
     1.
\end{enumerate}

\noindent Here are 3 examples you can enter on the command line that would
produce some of what you see in the table below:

\footnotesize
\begin {verbatim}
a v.fits	           (See Item A -- Changes the 
                     input file name to  v.fits.)

h 300 440 330 470	  (See Item H -- Changes the 
                     4 corners of the fitting box)

1 369.4 395.3 1 1   (Change the initial x,y position 
                     to (369.4,395.3), and allow both 
                     to vary)
\end {verbatim}
\normalsize

\noindent If you have more than one model and, for instance, you want to
change parameters for model {\it<number>} , you can do so at any time 
(before you start fitting) by first typing {\it m <number>} \noindent to 
designate it.  For example, to change the magnitude 
of object 4 to 19th magnitude, you have to first enter:

\begin {verbatim}
m 4
\end {verbatim}

\noindent followed by entering:

\begin {verbatim}
3 19.0     
\end {verbatim}

You only have to designate a component once if you're modifying that
component only.  Right now the program does not check to see if the 
parameters you entered make any sense or even if they're valid numbers.

\bigskip

\noindent {\bf Deleting an Object}\ \ \ \ \  To delete an object, use the 
{\it x} key, followed by the object number.  For example, 

\begin{verbatim}
x 3
\end{verbatim}

\noindent deletes the 3rd model from the fit.

\bigskip

\noindent {\bf Start Fitting}\ \ \ \ \ Once you're done with entering/changing 
all the parameters, to start fitting, hit ``q''.


\section {GALFIT MENU ITEMS}

This section describes the menu items (Figure 1) in a little more detail.  The
GALFIT menu is separated into two sections:  the image parameters (the top
half of the menu, i.e Items labeled A-S) and the object fitting parameters
(the bottom half, i.e. Items numbered 0-10 and Z).  There is only one set of
image parameters, but there can be an arbitrary number of object parameters
depending on the number of objects that you want to fit simultaneously.  In
principle there is no limit to the number of components, except by the
computer memory and the computation speed.

\subsection {The Image Parameters A-S}

{\bf Item A -- Input Data Image:} The input data image should be a FITS file
	 and it should be a single image, not an image block.  If the user
	 doesn't provide an input image, a model image will be created using
	 the input parameters.

\smallskip

{\bf Item B -- Output Image Block:} The output image block is a 3-D image cube
	 in FITS, created using CFITSIO (Pence 1999).  Starting from GALFIT
	 version 1.4h, the first image is blahblah.fits[1], the second one is
	 blahblah.fits[2], and the last is blahblah.fits[3].  The first image
	 is the postage stamp sized region specified in Item I.  Image [2] is
	 the final model of the galaxy in that region.  Image [3] is the
	 residual image by subtracting [2] from [1].

\smallskip

{\bf Item C -- Input Weight (Sigma) Image:} The sigma image, in FITS format,
	 is optional, and is used to give relative weights to the pixels
	 during the fit.  If one is not provided, or if the entry is ``none''
	 then a sigma image is generated from the data image (see \S~1.1).
	 The pixels in the weight image should have the same units as the data
	 image.  For example, if the image pixels are in electrons/sec then
	 the weight image should also have weights that are electrons/sec.
	 Otherwise, the $\chi^2$ that GALFIT outputs will not make much
	 sense.

\smallskip

{\bf Item D -- Convolution PSF, and (optional) CCD Diffusion Kernel:} The
	 observed PSF image, in FITS format, is required only if one want to
	 convolve a model with the PSF or fit a PSF to a star.  Otherwise, it
	 is optional.  The diffusion kernel is also optional, and is mostly
	 associated with {\it oversampled} HST PSFs that are created using the
	 TinyTim software for CCD imagers such as WFPC, WFPC2, STIS, and ACS
	 -- more details below.  If the PSFs are created through natural
	 stars, or if the TinyTim PSF is created to be 1-time oversampled, one
	 can ignore the CCD diffusion kernel.

	 The peak flux of the PSF image should be at the geometric center when
	 the number of pixels on a side is odd.  However, if the number of
	 pixels $N$ on a side is even, the peak should be located at pixel
	 position $(N/2 + 1)$.  If the peak is anywhere else, the model that
	 GALFIT generates will be systematically offset in position by the
	 difference from the predefined center.  This is important to keep in
	 mind when the convolution box is small:  you may want to make sure to
	 refit the image with a large convolution box after the solution has
	 first converged.

	 The input PSF can also be followed by an entry for the CCD charge
	 diffusion kernel, which is simply a text file.  It is usually only
	 needed if one has created an {\it oversampled} HST PSFs using
	 TinyTim.  If the PSF has unit sampling, the diffusion is applied by
	 TinyTim automatically, so GALFIT will not reapply it even if a kernel
	 is specified.  The appropriate charge diffusion kernel can be found
	 under the ``COMMENTS'' section in the PSF image header created by
	 TinyTim.  Here is an typical example of what the diffusion kernel
	 input file should look like:

\begin {verbatim}

   0.012500  0.050000  0.012500  
   0.050000  0.750000  0.050000  
   0.012500  0.050000  0.012500  

Table 1: An example of the WFPC2 CCD charge 
diffusion kernel:
\end {verbatim}

	 \noindent The charge diffusion convolution is applied only after the
	 model image has been convolved and rebinned (see next Item) down to
	 the original resolution of the HST imagers.  Note that if the science
	 data has been drizzled to a platescale other than the original
	 instrumental pixel scale, it makes no sense to apply the diffusion
	 kernel (see Item E for further explanation).  Also, note that
	 observed HST PSFs are sometimes slightly broader than the PSFs
	 generated by TinyTim, which may be caused by a small amount of
	 spacecraft jitter.

	 {\it NOTE:  The sky in the PSF image should be subtracted out, or
	 else in the model image there would appear to be a DC offset in the
	 values inside and outside the convolution box.  The PSF does not have
	 to normalized -- GALFIT will do so automatically.}  The DC offset
	 may also appear if the convolution box is small.

\smallskip

{\bf Item E -- PSF Oversampling Factor:} When the PSF is not Nyquist sampled
	 (FWHM $\gtrsim 2$ pixels), the Fourier transform is not uniquely
	 invertible.  Therefore, in principle all data should be Nyquist
	 sampled for convolution purposes.  Even if one might not have an
	 oversampled science image, sometimes one can obtain a reasonably
	 accurate oversampled PSF.  For HST data one way is to use the TinyTim
	 software.  The other alternative is to combine dithered sub-exposures
	 of bright stars during an observation, or by interpolating to get an
	 ``intrinsic'' PSF by using multiple stars in the same image, e.g.
	 with globular clusters.

	 As of version 2.0, GALFIT can deal with situations where the PSF is
	 more finely sampled than the data, i.e. {\it the PSF has a finer
	 pixel scale (arcsec/pix) than the data}, but both are observed under
	 the same seeing conditions.  You can think of it like observing a
	 field simultaneously with 2 cameras mounted on the same telescope,
	 one with a fine pixel scale (PSF camera) and the other with a coarse
	 scale (science camera).  When this happens, GALFIT will internally
	 produce models at the same sampling factor as the PSF, perform
	 convolution, and finally rebin the results to the data pixel scale.
	 Note that the oversampling factor can only be an integer value.

	 It is worth stressing that the oversampling factor is a {\it ratio
	 between the platescale (arcsec/pix) of the PSF and the DATA, observed
	 under the same seeing condition (i.e. optics convolved with
	 atmosphere)}.  If they have the same platescale + seeing, the factor
	 is always 1, no matter how finely they have been drizzled, dithered,
	 magnified, and combined from the original sub-exposures.

	 {\it A TECHNICAL MEMO REGARDING HST CCD DATA, i.e. for WFPC2, STIS,
	 WFPC, and ACS, but not NICMOS:} If the PSF being used is created from
	 natural stars, or if the TinyTim PSF is 1-time oversampled, there's
	 no need to apply the CCD diffusion kernel.  However, if the PSF is
	 created by TinyTim and N-times oversampled (N > 1), TinyTim assumes
	 the PSF will be binned down to the original CCD pixel scale, i.e.
	 {\it 1-time} oversampled, before applying the diffusion kernel.
	 However, GALFIT doesn't know anything about how the science data have
	 been drizzled and combined.  Suppose the original HST CCD data have
	 been drizzled to twice or more the original CCD resolution, then it
	 makes no sense to create an oversampled TinyTim PSF and use the
	 diffusion kernel that TinyTim provides, because GALFIT will bin the
	 PSF to the drizzled resolution, not the original CCD resolution.
	 However, GALFIT will apply the kernel blithely along.  The moral of
	 the story is that, for CCD data, if the science data have been
	 dithered and drizzled to a new resolution, either do not use a
	 TinyTim PSF directly, or else figure out a way to convolve the
	 TinyTim PSF with an appropriate kernel before using it in GALFIT, or
	 else figure out a way to transform the diffusion kernel to the new
	 platescale before giving the kernel to GALFIT.

\smallskip

{\bf Item F -- Bad Pixel Mask:} Sometimes one may want to exclude pixels from
	 a fit.  This bad pixel map can be either a FITS file or an ASCII text
	 file.  If the file is an ASCII, it should have 2 columns listing x
	 and y coordinates (without a comma separator).  If you want to mark
	 out an irregularly shaped region and have a list of polygon vertices,
	 you can run a program called ``fillpoly'' to create points inside the
	 polygon.  See GALFIT website on Frequently Asked Technical Questions
	 to get a copy.  The output file can then be read directly into Item
	 D.

	 If the dust map is a FITS image, the bad pixels should have a value
	 of $>0$.  while good pixels have a pixel value of 0.

\smallskip

{\bf Item G -- Parameter Coupling:} Parameter constraint/coupling file (ASCII)
	 is optional.  Parameters can be held fixed to within a certain range
	 or can be coupled between different components by providing this
	 file.  An example of the format is found in the file
	 EXAMPLE.CONSTRAINTS.  When constraints are imposed it is unclear what
	 the errorbars mean, if anything.  Furthermore, it may prematurely
	 force the solution to wander off into a corner of the parameter space
	 from which it is difficult to wander out.  So use constraints at own
	 risk!

\smallskip

{\bf Item H -- Fitting Region:} The image region to fit.  GALFIT will cut out
	 a section of the image specified by ``xmin xmax ymin ymax'' from the
	 original image and then minimize chi-square only over that region.
	 The fitting region should be large enough to include a significant
	 amount of background sky, if the sky is a free parameter in the fit.

\smallskip

{\bf Item I -- Convolution Box:} The convolution box size.  As of GALFIT
	 version 1.8b, the convolution boxes are centered on the individual
	 components, and follow them around so the user no longer has to
	 specify a convolution box {\it center}.  The convolution box can be
	 rectangular in size.

         The convolution box size and the PSF image size are the two most
	 important factors in determining how fast GALFIT will run.
	 Convolution can take up > 80\% of total execution time for a small
	 image, usually.  Because the convolution box follows around all
	 galaxy components, all components will be convolved at the center.
	 This is much more efficient than convolving the entire image.  Of
	 course, if one wants to convolve the entire image, one can still set
	 the convolution box to the fitting region size or even larger.

         In principle the box size should be just big enough so that the
	 seeing does not affect your galaxy profile outside of it (something
	 like 20 or more seeing diameters, depending on how extended the PSF
	 wing is.)

\smallskip

{\bf Item J -- Magnitude Zeropoint:} The magnitude zeropoint is used to
	 convert pixel values and fluxes into a physical magnitude by the
	 standard definition:

\begin{equation}
	{\rm mag} = -2.5 {\rm log}_{10} ({\rm ADUs} / t_{exp}) + {\mbox {mag zpt}}.
\end{equation}
\smallskip

	 \noindent The exposure time is taken from the EXPTIME image header
	 without prompting the user.  {\it So please make sure the EXPTIME
	 header reflects the data pixel values.} There are many images where
	 the EXPTIME may not reflect the fact the image ADUs has the exposure
	 time divided out.  If the ``EXPTIME'' keyword is not found, GALFIT
	 assumes the exposure time is 1 second.  If you want GALFIT to
	 generate a $\sigma_{x,y}$ image internally and the ADUs are in
	 [counts/sec], multiply EXPTIME back to the image and update the
	 EXPTIME header accordingly.

\smallskip

{\bf Item K -- Plate Scale:} The plate scale should be in units of arcseconds,
	 and is used only for the King and the Nuker profiles.  Note that the
	 sizes of objects ($r_e, r_s, r_b$, and FWHM) in the GALFIT output
	 files are in pixel units rather than in arcsec units.

\smallskip

{\bf Item O -- Interaction Window:} Sometimes it is nice to be able to quit
	 out of the fit in the middle and output the results or be able to
	 extend the number of maximum iterations (default = 100).  To do so,
	 you have the option of using the standard text window (regular, no
	 interaction), curses window (partial interaction), or both (which
	 opens a new green xterm).  I suggest using the ``both'' option (see
	 \S~5) whenever possible.

	 In the ``regular'' option there is no interaction possible.

	 In the ``both'' option, a green xterm window will pop up to show you
	 what commands one can issue to GALFIT while it is fitting.  To issue
	 commands, so so in the green window rather than in the fitting window
	 (except when the fit is paused, or when one is entering new number of
	 iterations).

	 In the ``curses'' mode, one can issue the same commands as the
	 ``both'' option.  But all the interaction is done in the fitting
	 window instead of a separate window.

	 {\it For Mac OS X users only:  To use the ``both'' mode, you must run
	 GALFIT in X11 (xterm) mode, otherwise it would complain about not
	 being able to open a window.  If not in X11, set Item O to
	 ``regular.''}

\smallskip

{\bf Item P -- Create Model:} If this option is set to 1, GALFIT will create
	 a model image based on your input parameters and immediately quit.

\smallskip

{\bf Item S -- Menu Interaction:} If Item S is set to 0, GALFIT starts fitting
	 immediately after the input file is read in.  It does not sit at the
	 menu prompt and wait for the user to hit ``Q'' to start fitting.


\subsection {Object Fitting Parameters 0-10, and Z}

GALFIT allows for a simultaneous fitting of arbitrary number of components
simply by extending the following object list (0-10, Z) for each component.
Items 1-10 are the initial rough guesses at the object parameters, and they
don't have to be accurate.  But of course, the more complicated a fit is, i.e.
the more number of components, the better the initial guesses should be so
GALFIT doesn't wander off to never-never-land.

The 2nd column in Items 3-10 are initial guesses for the parameter and the 3rd
column is where one can hold the parameters fixed (0) or allow them to vary
(1).  Note that items 1 and 2 ($x_c, y_c$) are on the same line (except
for fitting the sky).

Below is a more detailed explanation of what each of the parameters means:
 
{\bf Item 0:} Object name.  The valid entries are:  sersic, devauc, nuker,
	 expdisk, moffat, gaussian, sky, psf.

{\bf Items 1, 2:} X and Y positions of the galaxy in pixels.  For the sky,
	 Items 1 and 2 are on consecutive lines instead of the same line.  For
	 the sky Item 1 is the DC offset in {it ADUs}, and Item 2 is the
	 gradient in the X-direction.

{\bf Item 3:} For S\'ersic, de~Vaucouleurs, and exponential disk this is the
	 {\it integrated} magnitude of a galaxy.  For Nuker and the King
	 profile this is the {\it surface brightness} (mag/square arcsec)
	 calculated using the pixel scale from Item K.  For fitting the sky,
	 this is the sky gradient in the Y-direction.

{\bf Item 4:} Scalelengths of the fitted galaxy in PIXELS, not arcseconds.
	 The scale length is measured along the semi-major axis.

{\bf Item 5:} For S\'ersic it is the concentration index $n$.  For Nuker,
	 it is the powerlaw $\alpha$.  For King, it is the truncation
	 radius beyond which the fluxes are 0.  For all other functions
	 it is ignored.

{\bf Item 6:} For Nuker, it is the powerlaw $\beta$, and for King, it is
	 the powerlaw $\alpha$.  For all other functions, this parameter is
	 ignored.

{\bf Item 7:} For Nuker, it is the powerlaw $\gamma$.

{\bf Item 8:} The axis ratio is defined as semi-minor axis over the semi-major
	 axis: for a circle this value is 1, for an ellipse this value is less
	 than 1.

{\bf Item 9:} The position angle is 0 if the semi-major axis is aligned
	 parallel to the Y-axis and increases toward the counter-clockwise
	 direction.

{\bf Item 10:} The diskiness and boxiness parameter ($c$).  When $c<0$ the
	 ellipsoid appears disky, and when $c>0$, it appears boxy.

{\bf Item Z:} If you want this model to {\it not} be subtracted in the final
	 image, set this to 1.  If you want to subtract this model from the
	 data, set this to 0.  When this parameter is set to 0 for all the
	 objects, you will get a residual image.  The default is 0.

\section {THE GREEN POP-UP GALFIT WINDOW}

If Item O is set to ``both'', a green GALFIT window will pop up to indicate
that the fit is ongoing.  At any time, you can quit, output a menu file, pause
the fit, or change the number of iterations by hitting the keys ``q'',
``o'',``p'', or ``n'', respectively, {\it in the green pop-up window}.  At the
first convenient moment (i.e. after the current iteration), GALFIT will do
what you asked.  The pop-up window has memory, so if you hit a key multiple
times, e.g. out of frustration, it will be reissued at the earliest
convenience.  To get rid of the memory, kill the green window.

The output menu file (option ``o''is simply a pre-formatted file that you 
can feed back into GALFIT.  It stores parameters in the current fit iteration.

If you hit ``p'' or ``n'', you will receive a prompt in your iteration
window (not the green one).  You must answer the question in the iteration
window before it would go on.  Usually, GALFIT will decide when to quit
fitting on its own.  But, if GALFIT iterates over 100 times (an artificial
limit), it will also quit.  You can set the maximum number of iterations
you want this to happen by hitting ``n'' and specifying the number of
iterations.  Normally, GALFIT should converge between 10 to 30 iterations.

\section {OUTPUT FILES}

Once GALFIT finishes fitting, it will store the final parameter information
into 2 text files.  The first file, ``fit.log'' summarizes all the final
parameters and error bars for the fit.  This file just keeps getting appended
and does not get removed.  The errors quoted in ``fit.log'' are based on
diagonalizing and projecting the covariance matrix.  Thus the errors are
purely statistical -- some would say meaningless because the errors are often
dominated by systematics due to the real galaxies not being idealized
profiles.  But I don't know what to do about that.  The columns of numbers in
the output file are in the same order as the parameter numbers.

When GALFIT finishes fitting, it will also output a file called
``galfit.$NN$'' that contains all the best fit parameters.  $NN$ is a value
that keeps increasing so it will never overwrite the previous fit.  Note that
if there are missing numbers in the ``galfit.$NN$'' sequence, the gap will
eventually be filled.  One can modify this file and feed it back into GALFIT
to refine the fit.

Lastly, GALFIT will output a data image block, which is described in Item B of
\S~4.1.  The final fit parameters and a few important input parameters are
also placed into the FITS header of image[2] for convenience.

\section {HINTS ON DIFFICULT FITTING}

The above information is all that one needs to know to run GALFIT.  In
this section I will discuss some rules of thumb when dealing with
difficult situations.

For simple one or two component models GALFIT can usually perform without any
interaction.  But once in a while you may come across a galaxy GALFIT doesn't
like to fit and requires some massaging, e.g. if a galaxy is irregular, dusty,
or even sometimes when using the Nuker profile.  The Nuker profile can be a
little hard to use because the power law parameters are not intuitive, and can
be highly correlated.  When the program crashes or doesn't converge well, it
is often because the initial $\chi^2$ is too big (e.g. poor initial guesses,
irregular galaxy), or the powerlaws are too big or small (which can cause
numerical problems), or when GALFIT has settled into a local minimum (which
doesn't cause GALFIT to crash).  In troubled cases there are certain rules of
thumb that can help GALFIT along:

1.  One should consider using masking, especially if the area you're fitting
    is too dusty or irregular in shape.  For example, the jet in M87 will give
    GALFIT problems because the knots are bright, even though they are local
    features.  To create a mask, I have a program which can help, so if you
    need to mask a lot, please visit the "Frequently Asked Question" section
    on the GALFIT website.  There, I provide programs to expedite the process
    of creating masks.

2.  If a galaxy covering hundreds of pixels is hard to fit, try reducing the
    fitting region.  Once a good fit has been obtained, enlarge the fitting
    region.  I find this a handy trick for fitting many large galaxy images.

3.  If you can get a good estimate on the initial position of the galaxy peak,
    such as by eye or using IMEXAM, hold it fixed to that position.  If this
    is still not enough for the program to converge, hold the PA and
    ellipticity fixed at some reasonable guess values.  They don't have to be
    precise.  You can set them free later when the fitting problems go away.

4.  Also, for Nuker, make sure the parameters $\alpha, \beta, \gamma$, don't
    have the same values -- this can easily cause singular values, and force
    the program to quit.  Initially, your best bet for $\alpha$ and $\beta$
    parameters is somewhere between 0 and 3, and 0 to 1 for $\gamma$.  I've
    had something which converged from $\chi^2_\nu$ of $1\times10^6$ down to 1
    or 2 without causing the program to crash.  So the precise initial
    parameters aren't too important.

    To get a better feel for the behaviors of the parameters, take a look at
    Lauer et al. (1995) and Byun et al. (1997).

5.  Make sure the image header has the correct exposure time parameter
    (EXPTIME) and value.

6.  Hold some parameters fixed to allow more degenerate parameters, e.g.
    $\alpha, \beta, \gamma, I_e$, and $r_e$, to converge roughly, then let
    everything go free.

7.  If one wants to test whether a compact source is a true point source, one
    can convolve a narrow Gaussian with a PSF.  However, make sure FWHM > 0.3
    pixels or else the centroid may not move around much in position.  Usually
    it's a good idea to fit it once with a good FWHM $\sim$ 0.5 pixel.  Once
    the centroid is found, fit it again by allowing the FWHM to vary.
    However, this method is not the best way if one is interested in, say,
    extracting faint host galaxies from underneath luminous quasars.  The
    best way to do this is to fit a true point source.

I hope these 7 tips will help when fitting very difficult cases.  If the
suggestions above don't work after all of this, you're more than
welcome to:

8.  Email me (cyp@stsci.edu), and I'd be happy to give it a go to
    see what the problem might be.

\bigskip
\bigskip

Good luck Fitting!

\section {FREQUENTLY ASKED QUESTIONS and ADVISORY}

\noindent 1.  {\it ``Why is $\chi^2_\nu$ so small ($\ll 1\times10^{-2}$) or
   big ($\gg 1\times10^{3}$?)} for a good fit''\ \ \ \ \ The estimate of
   $\chi^2_\nu$ makes use of the sigma image.  For GALFIT to generate a
   reasonable sigma image, the image ADUs {\bf MUST} be in counts, {\bf NOT
   counts/sec}, with the appropriate GAIN, RDNOISE, and NCOMBINE in the image
   header.  If these things are not right and $\chi^2_\nu\ll 1$, the sigma
   image GALFIT generates is bunk.  GALFIT may quit prematurely when this
   happens, or produce a bad fit because the pixel weights are all screwed up.
   On the other hand,

\bigskip

\noindent 2.  If you are providing your own a sigma image, GALFIT doesn't care
   about the units of the image ADU as long as the sigma and image ADUs units
   are the same.

\bigskip

\noindent 3.  {\it ``What fitting region size should one use?''  Or, ``Why
    does the fit depend so sensitively on the fitting region?'' Or, ``Why does
    the fit depend so sensitively on the sky value?''}\ \ \ \ \ The fitting
    region should be considerably larger than the galaxy being fitted, if
    possible.  A common assumption is that the sky can be well determined if
    the image is 2, 3, or more, times the apparent size of a galaxy.  However,
    that may not be the case.  The extended wing of a galaxy is degenerate
    with the sky value.  Moreover, if the intrinsic galaxy profile is not the
    function being fitted, the sky parameter will always adjust to compensate.
    Thus, as a rule of thumb, try to determine the sky value independently and
    hold it fixed in the fit, and then compare the fit by allowing the sky to
    vary.  This will give you a good idea of how uncertain the fit is.

\bigskip

\noindent 3.  {\it ``When using the S\'ersic function why are the parameters
    of the fit so sensitive when $n$ is large ($n\gtrsim 3$)?''}\ \ \ \ \ When
    $n$ is large the profile has a steep central concentration and a highly
    extended wing that extends out farther than the eyes can see.  Because
    there are a lot more pixels at large radius, the fit is often dominated by
    the wings rather than by the central cusp.  Thus when there is a
    neighboring galaxy that is not masked out in the fit, $n$ can be driven
    high to try and fit its flux.  Also, because there is a strong correlation
    between $n$ and $R_e$, and $n$ with the background sky, a small change in
    $n$ or the sky can result in a large change in $R_e$ when $n$ is high.
    Therefore, be very careful about neighboring contamination and image size
    when the S\'ersic index $n$ is a high value (i.e.  $n>3$ or so).

\bigskip

\noindent 4. {\it ``The fit of a galaxy appears well centered, but why is there
    a systematic offset in the values of the position?''}\ \ \ \ \ The
    absolute position parameter depends on how well the convolution PSF is
    centered in the PSF image.  Nominally, the PSF should be centered on pixel
    $N/2+1$ for even $N$ number of pixels along a side, and $N/2+0.5$ for odd
    number of pixels.

\begin {references}

\reference {} 
Elson XXXX, ``Globular Clusters''.  Eds C. Martinez Roger, P.  Four\'non, F.
Sanchez.  Cambridge Contemp., Cambridge University Press.

\reference{}
Krist, J. E., \& Hook, R. N. 1997, in {\it HST}\ Calibration
Workshop with a New Generation of Instruments, eds. S. Casertano, et al.
(Baltimore: STScI), p. 192

\reference{} 
Lauer, T. R., Ajhar, E. A., Byun, Y.-I., Dressler, A., Faber, S.
M., Grillmair, C., Kormendy, J., Richstone, D., \& Tremaine, S. 1995, \aj,
110, 2622

\reference {}
Pence, W. 1999, ASP Conf. Ser., Vol. 172, 487

\reference {}
Peng, Ho, Impey, \& Rix 2002, \aj, 124, 266.

\reference {}
Press, W.~H., Teukolsky, S.~A., Vetterling, W.~T., \& Flannery, B.~P. 1997,
Numerical Recipes in C (Cambridge: Cambridge Univ.  Press)

\end{references}

\newpage

\footnotesize
\begin {verbatim}

# IMAGE PARAMETERS
A) galaxy.fits         # Input Data image (FITS file)
B) fit_out.fits        # Output data image block
C) none                # Noise image name (made from data if blank or "none") 
D) Rpsf.fits  kernel   # Input PSF image and (optional) diffusion kernel 
E) 2                   # PSF oversampling factor relative to data 
F) none                # Bad pixel mask (FITS image or ASCII coord list)
G) none                # File with parameter constraints (ASCII file) 
H) 133  380  133  380  # Image region to fit (xmin xmax ymin ymax)
I) 100    100          # Size of the convolution box (x y)
J) 22.08               # Magnitude photometric zeropoint 
K) 0.046  0.046        # Plate scale (dx dy). 
O) both                # Display type (regular, curses, both)
P) 0                   # Create output only? (1=yes; 0=optimize) 
S) 1                   # Modify/create objects interactively?

# INITIAL FITTING PARAMETERS
#
# Object type allowed: psf, sersic, devauc, nuker, expdisk, moffat, gaussian, sky.
#
# Column 2 :  Parameter value
# Column 3 :  Fit = 1, Fixed = 0
# Column 4+:  Parameter name

# Object number: 1 -- A TRUE point source
 0)        psf         #    Object type 
 1) 50.00  50.00  1 1    #  position x, y
 3) 18.000     1       #  total magnitude
 Z) 0                  #  Output option (0 = residual, 1 = Don't subtract) 

# Object number: 2
 0)     sersic         #    Object type
 1) 255.28  257.66  1 1  #   position x, y
 3) 14.00      1       #  total magnitude 
 4) 32.85      1       #      R_e
 5) 0.83       1       #  exponent (de Vaucouleurs = 4) 
 6) 0.00       0       #     ----- 
 7) 0.00       0       #     ----- 
 8) 0.87       1       #  axis ratio (b/a)  
 9) 6.22       1       #  position angle (PA) 
10) -0.17      1       #  diskiness(-)/boxiness(+)
 Z) 0                  #  Output image type (0 = residual, 1 = Don't subtract) 

# Object number: 3
 0)    expdisk         #    Object type
 1) 255.38  257.22  0 0  #   position x, y
 3) 12.98      0       #  total magnitude 
 4) 8.88       0       #      Rs 
 5) 0.00       0       #     ----- 
 6) 0.00       0       #     ----- 
 7) 0.00       0       #     ----- 
 8) 0.72       0       #  axis ratio (b/a)  
 9) 38.80      0       #  position angle (PA) 
10) -0.05      0       #  diskiness(-)/boxiness(+)
 Z) 0                  #  Output image type (0 = residual, 1 = Don't subtract) 

# Object number: 4
 0)      nuker         #    Object type
 1) 256.28  256.25  1 1  #   position x, y
 3) 15.88      1       #     mu(Rb)
 4) 36.13      1       #      Rb
 5) 1.17       1       #     alpha 
 6) 4.62       1       #      beta 
 7) 0.43       1       #     gamma 
 8) 0.72       1       #  axis ratio (b/a)  
 9) 38.97      1       #  position angle (PA) 
10) -0.04      1       #  diskiness(-)/boxiness(+)
 Z) 0                  #  Output image type (0 = residual, 1 = Don't subtract) 

# Object number: 5
 0)        sky         #    Object type 
 1) 2.00       0       #  sky background 
 2) 0.000      0       #  dsky/dx (sky gradient in x) 
 3) 0.000      0       #  dsky/dy (sky gradient in y) 
 Z) 0                  #  Output option (0 = residual, 1 = Don't subtract) 

\end{verbatim}

\figcaption[]  {Example of an input file.}

\newpage
\normalsize


\end {document}
